PREČKANJE PUŠČAVE NEZNANE VELIKOSTI

Povzetek
Klasična težava z avtom za prečkanje puščave šišre kot je naš domet avtomobila, se nanaša na problem, kjer vnaprej poznamo velikost puščave.
Rešitev in optimalne strategije za ta problem in njegove variacije so dobre znane, vendar je vedno predpostavljeno, da razdaljo vnaprej poznamo.
Mi rešujemo problem, ko razdalje ne poznamo vnaprej. Strategijo ocenjujemo po njenem konkurenčnem razmerju, ki je definiran kot razmerje med stroški najslabše možnega scenarija in stroški
optimalne rešitve, če bi razdaljo poznali vnaprej. Pokazali bomo/smo, da nobena strategija s fiksnim zaporedjem postaj, ne more doseči končnega konkurenčnega razmerja.
Optimalna strategija je iterativna strategija, ki uporablja optimalno strategijo za znano širino, da doseže ciljne razdalje in izprazni vse postaje med iteracijami.
Optimalna iterativna strategija podvoji stroške vsake uspešne zaporedne iteracije in doseže konkurenčno razmerje štiri.

UVOD
Imamo avtomobil z določeno prostornino goriva in določeno razdaljo, ki jo lahko prevozi s polnim tankom. BŠS lahko predpostavimo, da lahko avto nosi en liter goriva in, da 
lahko prevozi en kilometer na en liter goriva. Predpostavljamo, da je širina puščave večja kot en kilometer. Na začetku imamo neomejeno količino goriva, ki ga lahko poljubno shranimo
na poti. Naš cilj je minimanizirati stroške potovanja, ki jih merimo kot porabljeno količino goriva za dano razdaljo. To označimo kot zgornjo mejo prevožene razdalje in preostalo gorivo na postajah.
Pri enosmerni različici moramo prečkati dano puščavo, pri dvosmerni pa moramo biti sposobni se tudi vrnit na prvotno mesto.

PREJŠNJA DELA
Prve rešitve tega problema so bile predstavljene leta 1947 s strani Phipps, C. G. in Nathan Jacob Fine. Veliko lažje je določiti največjo razdaljo, ki jo lahko prevozimo z določeno količino goriva kot najmanjšo 
količino goriva, ki je potrebna za premagovanje določene razdalje. Fine je rešil enosmerni problem, Phipps pa dvosmernega in predstavil alternativno različico s karavano
avtomobilov, ki si lahko delijo gorivo, od katerih mora le eden dokončati potovanje. Druge različice tega problema so v [1,2,4-6].
Z enim litrom goriva lahko prevozimo $1/2$ kilometra in se vrnemo. Z dvema litroma lagko gremo $1/4$ kilometra, tam shranimo $1/2$ litra goriva in se vrnemo. 
Potem se peljemo $1/4$ kilometra, poberemo $1/4$ litra, se peljemo $1/2$ kilometra ter se vrnemo in poberemo preostalo $1/4$ litra in se vrnemo na začetek.
S tremi litri, lahko prevozimo $1/6 + 1/4 + 1/2$ kilometrov, s postajami na $1/6$ in $1/6 + 1/4$  kilometrih od začetka. V splošnem lahko z $n$ litri goriva 
prevozimo $1/2 + 1/4 + 1/6 + \dots + 1/(2n)$ kilometrov, z $n-1$ postajami na razdaljah $1/(2n), 1/(2n-1), \dots , 1/4$. Med vožnjo nikoli ne nosimo več goriva, 
kot je potrebno, da dosežemo začetek ali postajo. Vsak premik naprej začnemo s polnim tankom. Da bi našli najmanjšo količino goriva, potrebno za premaganje določene razdalje, 
poiščemo največjo vsoto tega zaporedja, ki je manjša ali enaka dani razdalji in shranimo goriva na teh lokacij od zadaj naprej. Potem pa naredimo toliko voženj, kolikor 
je potrebnih, da shranimo določeno količino goriva na prvi postjai. 
Količina goriva potrebnega, da prevozimo $d$ kilometrov in se vrnemo je $O(e^(2d))$ litrov [7].  Torej je cena goriva približno $7,389^d$ litrov, za velike $d$.

NEZNANA RAZDALJA IN KONKUREČNO RAZMERJE
Vsa prejšnja dela na tem problemu so temeljila na tem, da poznamo razdaljo, ki jo želimo premagat. Mi gledamo optimalno strategijo, kadar ne poznamo razdalje. 
Da ocenimo takšno optimalno strategijo, uporabimo konkurenčno razmerje. Tak pristop se uporablja, kadar se analizira algoritem z manjkajočimi ključnimi podatki za izračun.
Konkurenčno razmarje za takšnen problem je razmerje med stroškom za določeno razdaljo in med
optimalnim stroškom, če bi razdaljo poznali vnaprej. Najslabše konkurenčno razmerje je maksimum teh razmerij čez vse možne razdalje, optimalen algoritem pa je ta, kjer je ta 
maksimum najmanjši.

PREGLED
Na začetku pokažemo, da nobena strategija z enakomerno razdeljenimi postajami ne doseže končnega konkurenčnega razmerja. Nato pokažemo, da 
je možno vsako fiksno zaporedje postaj pretvoriti v enakomerno razporejene postaje, brez da bi izgubili na uspešnosti. Nato uvedemo iterativno strategijo, kjer avto prepotuje zaporedje vedno daljših razdalj 
z uporabo znane optimalne strategije za vsako razdaljo, kjer izprazni vse postaje med ponovitvami. Na koncu pokažemo, da eno optimalno zaporedje iteracij podvoji stroške vsake 
predhodne ponovitve in poveča razdaljo za malo manj kot tretjino kilometra vsako iteracijo.

STRATEGIJE S FISKNIM ZAPOREDJEM POSTAJ 
Najenostavnejša strategija bi fixno? zaporedno postavljala postaje, dokler ne pridemo do cilja. Da čim prej dosežemo cilj, si želimo čim dlje potovati, brez da bi zmanjšali učinkovitost. 
Za dano zaporedje postaj definiramo našo pot. Predpostavimo, da smo na postaji, s $p$ označimo razdaljo do predhodne postaje in skupno količino goriva v avtu in na postaji s $f$ litri.
Če velja $f - p \geq 1$, potem gremo naprej s polnim tankom. Če to ne velja, potem gremo nazaj s $p$ litri. Če smo na prvi postaji, štejemo kot prvo postajo začetek. S takšnim pristopom 
se vračamo z minimalno potrebno količino goriva, da pridemo do prve postaje in vedno nadaljujemo do začetka, naprej pa se premikamo s polnim tankom.   