\documentclass{beamer}

\usepackage{graphicx}
\usepackage{array}
\usepackage{tikz}
\usepackage[slovene]{babel}
\usepackage[utf8]{inputenc}
\usepackage[T1]{fontenc}
\usepackage{lmodern}

\usepackage{ulem}
\usepackage{url}
\usepackage{amsmath}
\usepackage[]{amsthm}

\usetheme{berlin}
\useinnertheme[shadows]{rounded}
\useoutertheme[]{infolines}
\beamertemplatenavigationsymbolsempty

\usepackage{palatino}
\usefonttheme{serif}

\title{Prečkanje puščave neznane velikosti}
\subtitle{Diplomski seminar}
\author[Justin Raišp]{Justin Raišp \\ \footnotesize Mentor: izr. prof. dr. David Dolžan}
\institute[FMF]{Fakulteta za matematiko in fiziko}
\date{21. 11. 2022}

\newtheorem{definicija}[]{Definicija}[]
\newtheorem{izrek}{Izrek}
\newtheorem{lema}{Lema}

%\qedhere

\begin{document}

% ===================================================================
\frame{\titlepage}

% -------------------------------------------------------------------
\section{Uvod}

% -------------------------------------------------------------------
\begin{frame}
   \frametitle{Predstavitev problema}
    Potrebno je prečkati puščavo neznane velikosti, pri čemer imamo na začetku na voljo neomejeno količino goriva, 
    vendar imamo končen rezervoar za gorivo v avtu. Brez škode za splošnost predpostavimo:
    \begin{itemize}
        \item rezervoar ima kapaciteto $1$ liter goriva,
        \item za $1$ kilometer potrebujemo $1$ liter goriva,
        \item poraba goriva je konstantna skozi celotno pot.
    \end{itemize}
    Zanima nas optimalna strategija postavljanja postaj z gorivom, da dosežemo cilj, pri čemer porabimo čim manjšo količino goriva.
\end{frame}
% ===================================================================
% -------------------------------------------------------------------
\begin{frame}
    \frametitle{Že rešeni problemi}
    Prečkanje puščave je v matematiki znan problem, ki se je prvič pojavil že v 9. stoletju, trenutna različica 
    problema pa se je pojavila v letu 1947 s strani Nathan Jacob Fine-a, vendar so ti problemi predpostavljali, da 
    poznamo širino dane puščave.
 \end{frame}
 % -------------------------------------------------------------------
 \section{Reševanje problema}

 % -------------------------------------------------------------------
 \begin{frame}
     \frametitle{Ideja}
     Z $n$ označimo količino goriva.
     \begin{itemize}
        \item $n = 1$: Peljemo se $\frac{1}{2}$ in se vrnemo na začetek
        \item $n = 2$: Peljemo se $\frac{1}{4}$, shranimo $\frac{1}{2}$ in se vrnemo. Nato se peljemo $\frac{1}{4}$, poberemo $\frac{1}{4}$, se peljemo 
        $\frac{1}{2}$, se vrnemo in po poti poberemo $\frac{1}{4}$,
        \item $n = 3$: Peljemo se $\frac{1}{6} + \frac{1}{4} + \frac{1}{2}$ in se vrnemo, vmes imamo $2$ postaji,
        \item[{$\vdots$}]
        \item $n = k$: Peljemo se $\frac{1}{2k} + \frac{1}{2(k-1)} + \dots + \frac{1}{4} + \frac{1}{2}$, vmes pa imamo $k - 1$ postaj.

     \end{itemize}

 \end{frame}
 
 % -------------------------------------------------------------------
 \begin{frame}
    \frametitle{Ideja}
    Ker imamo pot v obe smeri, pomeni da prevozimo  
    $$2\sum_{n=1}^{k} \frac{1}{2n} = \sum_{n=1}^{k} \frac{1}{n},$$
    kar nam da harmonično vrsto, katera divergira ko $k \to \infty$, torej lahko prevozimo 
    vsakršno razdaljo. $k$-to delno vsoto harmonične vrste lahko aproksimiramo z 
    $$\sum_{n=1}^{k} \frac{1}{n} \approx ln(k) + \gamma,$$ kjer je $\gamma \approx 0.577$ Euler-Macheronijeva konstanta.
    Torej lahko ocenimo, da za $d$ kilometrov in nazaj, potrebujemo $O(e^{2d})$ litrov goriva. Torej za velike $d$, je cena 
    goriva za razdaljo $d$ sorazmerna $7.389^d$.
\end{frame}

 % -------------------------------------------------------------------
 \begin{frame}

\end{frame}


\end{document}
