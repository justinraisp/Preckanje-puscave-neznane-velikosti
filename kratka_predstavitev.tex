\documentclass{beamer}

\usepackage{graphicx}
\usepackage{array}
\usepackage{tikz}
\usepackage[slovene]{babel}
\usepackage[utf8]{inputenc}
\usepackage[T1]{fontenc}
\usepackage{lmodern}

\usepackage{ulem}
\usepackage{url}
\usepackage{amsmath}
\usepackage[]{amsthm}

\usetheme{berlin}
\useinnertheme[shadows]{rounded}
\useoutertheme[]{infolines}
\beamertemplatenavigationsymbolsempty

\usepackage{palatino}
\usefonttheme{serif}

\title{Prečkanje puščave neznane velikosti}
\subtitle{Diplomski seminar}
\author[Justin Raišp]{Justin Raišp \\ \footnotesize Mentor: izr. prof. dr. David Dolžan}
\institute[FMF]{Fakulteta za matematiko in fiziko}
\date{21. 11. 2022}

\newtheorem{definicija}[]{Definicija}[]
\newtheorem{izrek}{Izrek}
\newtheorem{lema}{Lema}

%\qedhere

\begin{document}

% ===================================================================
\frame{\titlepage}

% -------------------------------------------------------------------
\section{Uvod}

% -------------------------------------------------------------------
\begin{frame}
   \frametitle{Predstavitev problema}
    Potrebno je prečkati puščavo neznane velikosti, pri čemer imamo na začetku na voljo neomejeno količino goriva, 
    vendar imamo končen rezervoar za gorivo v avtu. Brez škode za splošnost predpostavimo:
    \begin{itemize}
        \item rezervoar ima kapaciteto $1$ liter goriva,
        \item za $1$ kilometer potrebujemo $1$ liter goriva,
        \item poraba goriva je konstantna skozi celotno pot.
    \end{itemize}
    Zanima nas optimalna strategija postavljanja postaj z gorivom, da dosežemo cilj, pri čemer porabimo čim manjšo količino goriva.
\end{frame}
% ===================================================================
% -------------------------------------------------------------------
\begin{frame}
    \frametitle{Že rešeni problemi}
    Prečkanje puščave je v matematiki znan problem, ki se je prvič pojavil že v 9. stoletju, trenutna različica 
    problema pa se je pojavila v letu 1947 s strani Nathan Jacob Fine-a, vendar so ti problemi predpostavljali, da 
    poznamo širino dane puščave.
 \end{frame}
 % -------------------------------------------------------------------
 \section{Reševanje problema} 

 % -------------------------------------------------------------------
 \begin{frame}
     \frametitle{Ideja}
     Z $n$ označimo količino goriva.
     \begin{itemize}
        \item $n = 1$: Peljemo se $\frac{1}{2}$ in se vrnemo na začetek
        \item $n = 2$: Peljemo se $\frac{1}{4}$, shranimo $\frac{1}{2}$ in se vrnemo. Nato se peljemo $\frac{1}{4}$, poberemo $\frac{1}{4}$, se peljemo 
        $\frac{1}{2}$, se vrnemo in po poti poberemo $\frac{1}{4}$,
        \item $n = 3$: Peljemo se $\frac{1}{6} + \frac{1}{4} + \frac{1}{2}$ in se vrnemo, vmes imamo $2$ postaji,
        \item[{$\vdots$}]
        \item $n = k$: Peljemo se $\frac{1}{2k} + \frac{1}{2(k-1)} + \dots + \frac{1}{4} + \frac{1}{2}$, vmes pa imamo $k - 1$ postaj.

     \end{itemize}

 \end{frame}
 
 % -------------------------------------------------------------------
 \begin{frame}
    \frametitle{Ideja}
    Ker imamo pot v obe smeri, pomeni da prevozimo  
    $$2\sum_{n=1}^{k} \frac{1}{2n} = \sum_{n=1}^{k} \frac{1}{n},$$
    kar nam da harmonično vrsto, katera divergira ko $k \to \infty$, torej lahko prevozimo 
    vsakršno razdaljo. $k$-to delno vsoto harmonične vrste lahko aproksimiramo z 
    $$\sum_{n=1}^{k} \frac{1}{n} \approx ln(k) + \gamma,$$ kjer je $\gamma \approx 0.577$ Euler-Macheronijeva konstanta.
    Torej lahko ocenimo, da za $d$ kilometrov in nazaj, potrebujemo $O(e^{2d})$ litrov goriva. Torej za velike $d$, je cena 
    goriva za razdaljo $d$ sorazmerna $7.389^d$.
\end{frame}

 % -------------------------------------------------------------------
 \begin{frame}
    \frametitle{Učinkovitost strategije}
    Učinkovitost strategije ocenimo s "worst case competitive ratio", ki je značilen za ocenjevanje učinkovitosti pri 
    problemih, kjer ključne informacije niso znane vnaprej. Definiran je kot: 
    $$\frac{\text{Najvišja cena z danim algoritmom}}{\text{Optimalna cena, če poznamo razdaljo apriori}} $$
\end{frame}

 % -------------------------------------------------------------------
 \begin{frame}
    \frametitle{Enakomerno razporejene postaje}
    Najpreprostejši način reševanja tega problema bi bile enakomerno razporejene postaje z gorivom. Ideja je, da se
    premikamo naprej dokler je možno. Definiramo \textbf{razvrščanje vnaprej}: predpostavimo, da smo na postaji in s $p$ 
    označimo razdaljo do predhodne postaje in trenutno količino goriva, vključno z gorivmo na postaji s $f$. Če velja 
    $f - p \geq 1$, gremo naprej s polnim rezervoarjem. Sicer se vrnemo s $p$ litri goriva. 
\end{frame}


 % -------------------------------------------------------------------
 \begin{frame}
    \frametitle{Enakomerno razporejene postaje}
    Za primer vzamemo enakomerno razporejene postaje na razdalji $\frac{1}{3}$.
    \begin{itemize}
        \item Povratna pot do prve postaje nas stane $\frac{1}{3}$ v vsako smer, vmes pa shranimo $\frac{1}{3}$ goriva,
        \item Dve takšni poti nam omogočita $\frac{2}{3}$ goriva na prvi postaji,
        \item Na tretji poti porabimo $\frac{1}{3}$ do prve postaje, vzamemo $\frac{1}{3}$, se peljemo do 
                druge postaje, odložimo $\frac{1}{3}$, in se vrnemo na začetek z vmesno postajo na $\frac{1}{3}$.
    \end{itemize}

\end{frame}


 % -------------------------------------------------------------------
 \begin{frame}
    \frametitle{Enakomerno razporejene postaje}
    Z $f(n)$ označimo začetno količino goriva, potrebnega za dostavo $\frac{1}{3}$ goriva do $n$-te 
    postaje, s predpostavko, da so vse vmesne postaje prazne. Potem velja
    \begin{itemize}
        \item $f(1) = 1$,
        \item $f(2) = 2 f(1) + 1 = 3$,
        \item $f(3) = 2 f(2) + 2 f(1) + 1 = 9$
        \item[{$\vdots$}]
    \end{itemize}
    Za $n$-to postajo dobimo formulo:
        $$f(n) = 1 + \sum_{i=1}^{n-1} 2f(i) = 3^{n-1}$$
    Torej, če imamo $3$ postaje na kilometer, nas povratna pot dolžine $d$ stane $3^{3d-1} = O(27^d)$.
\end{frame}


 % -------------------------------------------------------------------
 \begin{frame}
    \frametitle{Enakomerno razporejene postaje}
    Če postavimo postaje na razdalji  $\frac{1}{4}$ , nas povratna pot dolžine $d$ stane $O(16^d)$,
    kar je veliko boljše kot $O(27^d)$, vendar veliko slabše kot $O(7.389^d)$, kar dobimo pri optimalni strategiji
    pri znani razdalji. Zaradi večje eksponentne rasti teh strategij, je njihov konkurenčni kriterij neomejen.
    Da se pokazati, da pri nobeni strategiji z enakomerno razporejenimi postajami ne moremo doseči končnega 
    konkurenčnega razmerja.
\end{frame}


 % -------------------------------------------------------------------
 \begin{frame}
    \frametitle{Enakomerno razporejene postaje}
    \begin{izrek}
        Pri razvrščanju naprej z enakoremo razporejenimi postajami $\frac{1}{k}$ narazen, asimptotična
        cena za doseganje dane razdalje raste eksponentno z razdaljo, in osnova eksponenta je omejena od spodaj z
         $(\frac{k}{k-2})^k$
    \end{izrek}
\end{frame}

 % -------------------------------------------------------------------
 \begin{frame}
    \frametitle{Optimalnost}
    Da se pokazati, da je razvrščanje vnaprej optimalna strategija. Vendar če imamo enakomerno razporejene 
    postaje na razdaljah $\frac{1}{k}$, lahko učinkovitost povečamo, če dodamo na sredino vsakih dveh postaj novo postajo. 
    Posledično z nobeno strategijo z enakomerno razporejenimi postajami za neznano razdaljo, ne dosežemo končnega "konkurenčno 
    razmerje v najslabšem primeru", saj je edini način da dosežemo stopnjo $e^{2}$, da $k$ pošljemo v neskončnost, kar bi 
    pomenilo, da bi potrebovali neskončno količino goriva za katerokoli razdaljo od začetka.
    \end{frame}

% -------------------------------------------------------------------
 \begin{frame}
    \frametitle{Iteracija}
    V nadaljevanju se bom ukvarjal z iterativnim pristopom k temu problemu in kako s pomočjo dinamičnega programiranja  
    določimo "konkurenčno razmerje v najslabšem primeru".
   \end{frame}

% -------------------------------------------------------------------   
\section{Zaključek} 
% -------------------------------------------------------------------
 \begin{frame}
    \frametitle{Literatura}
    \begin{thebibliography}{9}
        \bibitem{A jeep crostting a desert of unknown width}
        Richard E. Korf (2022) \textit{A Jeep Crossing a Desert of Unknown Width}, The
        American Mathematical Monthly, 129:5, 435-444, DOI: 10.1080/00029890.2022.2051404
                
        \bibitem{Harmonic series}
        Weisstein, Eric W, \textit{Harmonic Series}, [ogled 11. 11. 2022], dostopno na \url{https://mathworld.wolfram.com/HarmonicSeries.html}
        \end{thebibliography}
   \end{frame}
\end{document}
