\documentclass{beamer}


\mode<presentation>
{
  \usetheme{Warsaw}
  % or ...

  \setbeamercovered{transparent}
  % or whatever (possibly just delete it)
}

\usepackage{booktabs}
\usepackage{graphicx}
\usepackage{array}
\usepackage{tikz}
\usepackage[slovene]{babel}
\usepackage[utf8]{inputenc}
\usepackage[T1]{fontenc}
\usepackage{lmodern}

\usepackage{ulem}
\usepackage{url}
\usepackage{amsmath}
\usepackage[]{amsthm}

%\usetheme{berlin}
%\useinnertheme[shadows]{rounded}
\useoutertheme[]{infolines}
\beamertemplatenavigationsymbolsempty
\setbeamertemplate{theorems}[numbered]
\usepackage{palatino}
\usefonttheme{serif}


\makeatletter
\newenvironment<>{proofs}[1][\proofname]{%
    \par
    \def\insertproofname{#1\@addpunct{.}}%
    \usebeamertemplate{proof begin}#2}
  {\usebeamertemplate{proof end}}
\makeatother


\title{Prečkanje puščave neznane velikosti}
\subtitle{Diplomski seminar}
\author[Justin Raišp]{Justin Raišp \\ \footnotesize Mentor: izr. prof. dr. David Dolžan}
\institute[FMF]{Fakulteta za matematiko in fiziko}
\date{13. 4. 2023}

\newtheorem{definicija}[]{Definicija}[]
\newtheorem{izrek}{Izrek}
\newtheorem{lema}{Lema}
\newtheorem{dokaz}{Dokaz}

%\qedhere

\begin{document}

% ===================================================================
\frame{\titlepage}


% -------------------------------------------------------------------
\section{Uvod}

% -------------------------------------------------------------------
\begin{frame}
   \frametitle{Predstavitev problema}
    Potrebno je prečkati puščavo neznane velikosti, pri čemer imamo na začetku na voljo neomejeno količino goriva, 
    vendar imamo končen rezervoar za gorivo v avtu. Brez škode za splošnost predpostavimo:
    \begin{itemize}
        \item rezervoar ima kapaciteto $1$ liter goriva,
        \item za $1$ kilometer potrebujemo $1$ liter goriva,
        \item poraba goriva je konstantna skozi celotno pot.
    \end{itemize}
    Zanima nas optimalna strategija postavljanja postaj z gorivom, da dosežemo cilj, pri čemer porabimo čim manjšo količino goriva.
\end{frame}
% ===================================================================
% -------------------------------------------------------------------
\begin{frame}
    \frametitle{Pojavitev problema}
    Prečkanje puščave je v matematiki znan problem, ki se je prvič pojavil že v 9. stoletju, trenutna različica 
    problema pa se je pojavila v letu 1947 s strani Nathan Jacob Fine-a, vendar so ti problemi predpostavljali, da 
    poznamo širino dane puščave.
 \end{frame}
 % ------------------------------------------------------------------- 

 % -------------------------------------------------------------------
 \begin{frame}
     \frametitle{Ideja}
     Z $n$ označimo količino goriva.
     \begin{itemize}
        \item $n = 1$: Peljemo se $\frac{1}{2}$ in se vrnemo na začetek
        \item $n = 2$: Peljemo se $\frac{1}{4}$, shranimo $\frac{1}{2}$ in se vrnemo. Nato se peljemo $\frac{1}{4}$, poberemo $\frac{1}{4}$, se peljemo 
        $\frac{1}{2}$, se vrnemo in po poti poberemo $\frac{1}{4}$,
        \item $n = 3$: Peljemo se $\frac{1}{6} + \frac{1}{4} + \frac{1}{2}$ in se vrnemo, vmes imamo $2$ postaji,
        \item[{$\vdots$}]
        \item $n = k$: Peljemo se $\frac{1}{2k} + \frac{1}{2(k-1)} + \dots + \frac{1}{4} + \frac{1}{2}$, vmes pa imamo $k - 1$ postaj.

     \end{itemize}

 \end{frame}
 
 % -------------------------------------------------------------------
 \begin{frame}
    \frametitle{Ideja}
    Ker imamo pot v obe smeri, pomeni da prevozimo  
    $$2\sum_{n=1}^{k} \frac{1}{2n} = \sum_{n=1}^{k} \frac{1}{n},$$
    kar nam da harmonično vrsto, katera divergira ko $k \to \infty$, torej lahko prevozimo 
    vsakršno razdaljo. $k$-to delno vsoto harmonične vrste lahko aproksimiramo z 
    $$\sum_{n=1}^{k} \frac{1}{n} \approx ln(k) + \gamma,$$ kjer je $\gamma \approx 0.577$ Euler-Macheronijeva konstanta.
    Torej lahko ocenimo, da za $d$ kilometrov in nazaj, potrebujemo $O(e^{2d})$ litrov goriva. Torej za velike $d$, je cena 
    goriva za razdaljo $d$ sorazmerna $7.389^d$. V primeru znane razdalje, je to optimalna rešitev.
\end{frame}


 % -------------------------------------------------------------------
 \begin{frame}
    \frametitle{Učinkovitost strategije}
    Učinkovitost strategije ocenimo s "konkurenčnim razmerjem v najslabšem primeru", ki je značilen za ocenjevanje učinkovitosti pri 
    problemih, kjer ključne informacije niso znane vnaprej. Definiran je kot: 
    $$\frac{\text{Najvišja cena z danim algoritmom glede na vse možne razdalje}}{\text{Optimalna cena, če poznamo razdaljo apriori}} $$
    V našem primeru je za velike razdalje $d$ optimalna cena približno $7.389^d$.
\end{frame}


 % -------------------------------------------------------------------
 \begin{frame}
    \frametitle{Razvrščanje vnaprej}
    Definiramo \textbf{razvrščanje skokov vnaprej}: predpostavimo, da smo na postaji in s $p$ 
    označimo razdaljo do predhodne postaje in trenutno količino goriva, vključno z gorivom na postaji s $f$. Če velja 
    $f - p \geq 1$, gremo naprej s polnim rezervoarjem. Sicer se vrnemo s $p$ litri goriva. Ideja je, da se
    premikamo naprej dokler je možno. Definiramo \textit{skok} kot premikanje med dvema sosednjima postajama in
    \textit{izlet} kot pot od začetka do določene postaje in nazaj.
\end{frame}


 % -------------------------------------------------------------------
 \begin{frame}
    \frametitle{Enakomerno razporejene postaje}
    Najpreprostejši način reševanja tega problema bi bile enakomerno razporejene postaje z gorivom.  \\
    Za primer vzamemo enakomerno razporejene postaje na razdalji $\frac{1}{3}$.
    \begin{itemize}
        \item Povratna pot do prve postaje nas stane $\frac{1}{3}$ v vsako smer, vmes pa shranimo $\frac{1}{3}$ goriva,
        \item Dve takšni poti nam omogočita $\frac{2}{3}$ goriva na prvi postaji,
        \item Na tretji poti porabimo $\frac{1}{3}$ do prve postaje, vzamemo $\frac{1}{3}$, se peljemo do 
                druge postaje, odložimo $\frac{1}{3}$, in se vrnemo na začetek z vmesno postajo na $\frac{1}{3}$.
    \end{itemize}
\end{frame}


 % -------------------------------------------------------------------
 \begin{frame}
    \frametitle{Enakomerno razporejene postaje}
    Z $f(n)$ označimo začetno količino goriva, potrebnega za dostavo $\frac{1}{3}$ goriva do $n$-te 
    postaje, s predpostavko, da so vse vmesne postaje prazne. Potem velja
    \begin{itemize}
        \item $f(1) = 1$,
        \item $f(2) = 2 f(1) + 1 = 3$,
        \item $f(3) = 2 f(2) + 2 f(1) + 1 = 9$
        \item[{$\vdots$}]
    \end{itemize}
    Za $n$-to postajo dobimo formulo:
        $$f(n) = 1 + \sum_{i=1}^{n-1} 2f(i) = 3^{n-1}$$
    Torej, če imamo $3$ postaje na kilometer, nas povratna pot dolžine $d$ stane $3^{3d-1} = O(27^d)$.
\end{frame}


 % -------------------------------------------------------------------
 \begin{frame}
    \frametitle{Enakomerno razporejene postaje}
    Če postavimo postaje na razdalji  $\frac{1}{4}$ , nas povratna pot dolžine $d$ stane $O(16^d)$,
    kar je veliko boljše kot $O(27^d)$, vendar veliko slabše kot $O(7.389^d)$, kar dobimo pri optimalni strategiji
    pri znani razdalji. Zaradi večje eksponentne rasti teh strategij, je njihov konkurenčni kriterij neomejen.
    Da se pokazati, da pri nobeni strategiji z enakomerno razporejenimi postajami ne moremo doseči končnega 
    konkurenčnega razmerja.
\end{frame}


 % -------------------------------------------------------------------
 \begin{frame}
    \frametitle{Enakomerno razporejene postaje}
    \begin{izrek}
        Pri razvrščanju vnaprej z enakomerno razporejenimi postajami $\frac{1}{k}$ narazen, asimptotična
        cena za doseganje dane razdalje raste eksponentno z razdaljo, in osnova eksponenta je omejena od spodaj z
         $(\frac{k}{k-2})^k$
    \end{izrek}
\end{frame}


 % -------------------------------------------------------------------
 \begin{frame}
    \frametitle{Razvrščanje vnaprej}
    \begin{lema}
        Razvrščanje skokov vnaprej je optimalna strategija.
    \end{lema}
\end{frame}


 % -------------------------------------------------------------------
 \begin{frame}
    \frametitle{Razvrščanje vnaprej}
    \begin{Proof}
        Pogledamo poljubne tri sosednje postaje $x$, $y$ in $z$, pri čemer je $x$ najbližja začetku. Pri poljubni rešitvi danega problema, bo določeno število skokov med $x$ in 
        $y$ ter med $y$ in $z$. V optimalni rešitvi se bo vsak skok začel s polnim rezervoarjem. Število skokov je določeno s količino goriva potrebnega v $z$. 
        Neodvisno od zaporedja skokov je končno število skokov od $x$ do $y$ in od $y$ do $z$ enako $\Leftrightarrow$ v $y$ je vedno dovolj goriva, da potujemo do $z$ s polnim rezervoarjem. 
        Razvrščanje vnaprej upošteva to omejitev, torej je optimalna strategija.
    \end{Proof}
\end{frame}


% -------------------------------------------------------------------
\begin{frame}
    \frametitle{Enakomerno razporejene postaje}
    \begin{definicija}
        Učinkovitost prenosa goriva je definirana kot količina goriva dodanega k zadnji postaji, pomnožena z razdaljo od izhodišča do cilja, deljena s 
        količino goriva, ki smo jo odvzeli iz izhodišča.

    \end{definicija}
    \begin{lema}
        Pri razvrščanju vnaprej, za katerikoli sosednji postaji velja, da lahko vedno izboljšamo učinkovitost 
        tako, da med njima postavimo novo postajo. Optimalna pozicija za novo postajo je na sredini med danima 
        postajama.
    \end{lema}
    
\end{frame}


% -------------------------------------------------------------------
\begin{frame}
    \frametitle{Enakomerno razporejene postaje}
    \begin{lema}
        V primeru neznane razdalje bodo postaje v optimalni rešitvi med seboj enako oddaljene.
    \end{lema}
\end{frame}


% -------------------------------------------------------------------
\begin{frame}
    \frametitle{Enakomerno razporejene postaje}
    \begin{Proof}
        Predpostavimo, da imamo optimalno rešitev z neenakomerno postavljenimi postajami. Potem obstajajo $x, y, z$, kjer je razdalja med $x$ in $y$ 
        različna od razdalje med $y$ in $z$. Prejšnja lema nam pove, da se da izboljšati učinkovitost tako, da $y$ prestavimo na sredino med $x$ in $z$. 
        Torej se našo optimalno rešitev da izboljšati in dobimo protislovje.
    \end{Proof}
\end{frame}


% -------------------------------------------------------------------
\begin{frame}
    \frametitle{Enakomerno razporejene postaje}
    \begin{izrek}
        Nobena strategija za neznano razdaljo s fiksnim zaporedjem postaj ne more doseči končnega konkurenčnega razmerja.
    \end{izrek}
\end{frame}


% -------------------------------------------------------------------
\begin{frame}
    \frametitle{Enakomerno razporejene postaje}
    \begin{Proof}
        Predpostavimo, da imamo optimalno strategijo za neznano razdaljo s fiksnim zaporedjem postaj. Lema 2 nam pove, da lahko 
        razvrstimo te postaje z razvrščanjem vnaprej, brez da bi izgubili učinkovitost. Lema 4 nam pove, da morajo biti postaje enakomerno 
        oddaljene med seboj. Izrek 1 nam pa pove, da se stroški razvrščanja vnaprej s postajami oddaljenimi $\frac{1}{k}$ večajo eksponentno. 
        Ker je osnova eksponenta za to strategijo vsaj $(\frac{k}{k-2})^k$ in za $\forall k > 2$ velja: 
        $$ \left (\frac{k}{k-2} \right )^k > e^2,$$
        dobimo neomejeno konkurenčno razmerje v najslabšem primeru.

    \end{Proof}
\end{frame}


 % -------------------------------------------------------------------
 \begin{frame}
    \frametitle{Optimalnost}
    Da se pokazati, da je razvrščanje vnaprej optimalna strategija. Vendar če imamo enakomerno razporejene 
    postaje na razdaljah $\frac{1}{k}$, nam lema pokaže, da lahko učinkovitost povečamo, če dodamo na sredino vsakih dveh postaj novo postajo. 
    Posledično z nobeno strategijo z enakomerno razporejenimi postajami za neznano razdaljo, ne dosežemo končnega "konkurenčnega
    razmerja v najslabšem primeru", saj je edini način da dosežemo stopnjo $e^{2}$, da $k$ pošljemo v neskončnost, kar bi 
    pomenilo, da bi potrebovali neskončno količino goriva za katerokoli razdaljo od začetka.
    \end{frame}

% -------------------------------------------------------------------
 \begin{frame}
    \frametitle{Iteracija}
    Če zaporedje postaj ni fiksno, se mora skozi potovanje spreminjati. Nesmiselno je spreminjati postaje, ki so dlje od naše trenutne točke, saj bi bilo enako, če bi 
    bile spremenjene od začetka. Če dodamo nove postaje na že prevoženo pot, nam lema 2 pove, da je bolj učinkovito, če vključimo te postaje že na začetku. Podobno velja za odstranjevanje 
    postaj. Edina preostala možnost je, da delamo iteracije, s katerimi dosežemo vedno daljšo razdaljo, pri čemer odstranimo vso gorivo predhodnih iteracij.
   \end{frame}


% -------------------------------------------------------------------
\begin{frame}
    \frametitle{Iteracija}
    Za $n$-to iteracijo definiramo konkurenčno razmerje, povezano z iskanjem cilja v vsaki iteraciji kot:
    $$ \frac{\text{Vsota cen prvih $n$ iteracij}}{\text{Cena $(n-1)$-te iteracije}}$$
    \begin{lema}
        Obstaja optimalna iterativna strategija, kjer so konkurenčna razmerja povezana z iskanjem cilja v vsaki iteraciji, enaka.
    \end{lema}
   \end{frame}


% -------------------------------------------------------------------
\begin{frame}
    \frametitle{Iteracija}
    \begin{lema}
        V iterativni strategiji z enakimi konkurenčnimi razmerji imamo padajoče zaporedje multiplikatorjev, od $m_2$ dalje.
    \end{lema}
   \end{frame}


% -------------------------------------------------------------------
\begin{frame}
    \frametitle{Iteracija}
    \begin{izrek}
        Ena izmed optimalnih strategij, merjene glede na konkurenčno razmerje v najslabšem primeru, je iterativna 
        strategija, ki podvoji strošek vsake zaporedne iteracije in ima konkurenčno razmerje enako $4$.
    \end{izrek}
   \end{frame}

   
% -------------------------------------------------------------------
\begin{frame}
    \frametitle{Iteracija}
    \begin{proofs}
        Pokazali smo, da je edina optimalna alternativa fiksnemu zaporedju postaj iterativna strategija, ki uporablja
        optimalno strategijo za znano razdaljo, za doseganje vedno večjih razdalj. Lema 4 nam pove, da če obstaja optimalna iterativna 
        strategija, potem imamo optimalno strategijo, kjer so vsa konkurenčna razmerja enaka. Lema 5 nam pove, da imamo v tem primeru 
        padajoče zaporedje multiplikatorjev iteracij. Multiplikatorji morejo biti vsi večji od $1$, saj mora vsaka iteracija premagati večjo razdaljo. 
        Torej mora zaporedje multiplikatorjev konvergirati k limiti, ki je večja ali enaka $1$. Konkurenčno razmerje za dano iteracijo lahko zapišemo kot 
        vsoto členov, ki vsebujejo multiplikator za to iteracijo, število $1$ ter zaporedje, kjer so členi recipročne vrednosti produkta vedno daljšega zaporedja multiplikatorjev.
    \end{proofs}
    \end{frame}


% -------------------------------------------------------------------
\begin{frame}
    \frametitle{Iteracija}
    \begin{proofs}
        Upoštevamo tri primere: 
        \begin{enumerate}
            \item multiplikatorji konvergirajo k $1$ in njihov produkt konvergira h končnemu številu,
            \item multiplikatorji konvergirajo k $1$ in njihov produkt je neomejen,
            \item multiplikatorji konvergirajo k številu večjemu od $1$, torej je njihov produkt neomejen.
        \end{enumerate}
        V prvem primeru recipročne vrednosti produkta konvergirajo k neničelnemu številu. Konkurenčna razmerja so vsota teh recipročnih vrednosti in, ker je lahko neskončno število 
        iteracij, bo konkurenčno razmerje neomejeno. To pomeni, da to ni optimalno, saj lahko dosežemo končno razmerje. 

    \end{proofs}
   \end{frame}



% -------------------------------------------------------------------
\begin{frame}
    \frametitle{Iteracija}
    \begin{Proof}
        V drugem in tretjem primeru produkti rastejo čez vse meje, torej 
        gredo njihove recipročne vrednosti proti ničli. Torej se členi, ki vključujejo začetne multiplikatorje, bližajo proti nič in njihovo konkurenčno razmerje 
        je določeno z limito multiplikatorjev $m$. \\
        Konkurenčno razmerje je vsota $m + 1 + \frac{1}{m} + \frac{1}{m^2} + \frac{1}{m^3} + \dots$. Ker vemo, da je $m > 1$, vrsta 
        $\sum_{i=0}^{\infty} m^{-i}$ konvergira in dobimo konkurenčno razmerje kot $$m + \sum_{i=0}^{\infty} m^{-i} = m + \frac{m}{m-1} = \frac{m^2}{m-1}.$$
        Ta izraz odvajamo in enačimo z nič in dobimo vrednost $m = 2$ ter naše konkurenčno razmerje, ki je enako $4$.
    \end{Proof}
\end{frame}

% -------------------------------------------------------------------
\begin{frame}
    \frametitle{Iteracija}
    Želimo z vsako iteracijo podvojit ceno prejšnje iteracije in nas zanima koliko se v tem primeru podaljša naša pot glede na prejšnjo iteracijo. \\
    Za doseganje razdalje $d$ potrebujemo $O(e^{2d})$ goriva, torej za doseganje
    $x + d$ potrebujemo $O(e^{2(d+x)})$. To nam da enačbo: 
    $$ 2e^{2d} = e^{2(d+x)} \Leftrightarrow x = \frac{ln2}{2} \approx 0.3465736$$
    Torej je ena izmed optimalnih strategij niz iteracij, kjer je vsaka iteracija za nekaj več kot tretjino daljša 
    od prejšnje.
\end{frame}

% -------------------------------------------------------------------
\begin{frame}
    \frametitle{Prvih 12 iteracij strategije podvajanja}
    \begin{table}
        \centering
        \begin{tabular}{cccccc}
        \toprule
        Iter. & Gorivo & Razdalja & Dodana razd. & Konk. raz. & Raz. goriva \\
        \midrule
        1 & 1 & 0.5 & & &  \\
        2 & 2 & 0.75 & 0.25 & 2.5 & 2 \\
        3 & 4 & 1.041667 & 0.291667 & 3.208333 & 2 \\
        4 & 8 & 1.358929 & 0.317262 & 3.591369 & 2 \\
        5 & 16 & 1.690364 & 0.331435 & 3.792141 & 2 \\
        6 & 32 & 2.029248 & 0.338884 & 3.89514 & 2 \\
        7 & 64 & 2.371945 & 0.342697 & 3.947331 & 2 \\
        8 & 128 & 2.716574 & 0.344629 & 3.973605 & 2 \\
        9 & 256 & 3.062172 & 0.345598 & 3.986788 & 2 \\
        10 & 512 & 3.408258 & 0.346086 & 3.99339 & 2 \\
        11 & 1024 & 3.754588 & 0.34633 & 3.996694 & 2 \\
        12 & 2048 & 4.101039 & 0.346451 & 3.998347 & 2 \\
        \bottomrule
        \end{tabular}
        \end{table}
\end{frame}


% -------------------------------------------------------------------
\begin{frame}
    \frametitle{Optimalna začetna iteracija}
    Preprosto strategijo podvajanja lahko nekoliko izboljšamo tako, da z vsako novo iteracijo dosežemo nekoliko daljšo razdaljo kot $\frac{ln2}{2}$ in še vedno dosežemo najslabše konkurenčno razmerje 4. Da izračunamo konkurenčno razmerje, 
    ocenimo stroške iteracije kot skupno prevoženo razdaljo do cilja in nazaj, namesto količino goriva odstranjenega z začetka.
    
\end{frame}

% -------------------------------------------------------------------
\begin{frame}
    \frametitle{Optimalna začetna iteracija}
    \begin{itemize}
        \item Najdaljša prva iteracija s konkurenčnim razmerjem 4 ali manj porabi $8l$, da doseže $1,3589 km$. V najslabšem primeru je cilj tik za $5.$ postajo, $0,44226km$ od izhodišča.
        \item Druga iteracija porabi $25l$, da doseže $1,908km$. Ker je to več kot $1/2km$ dlje kot prejšnja, ta iteracija ne doseže zadnje postaje, s čimer prihrani nekaj več kot liter goriva. In ker velja 
                $(8+24)/8 = 4$, je konkurenčno razmerje te iteracije manjše od $4$.
    \end{itemize}

\end{frame}

% -------------------------------------------------------------------
\begin{frame}
    \frametitle{Optimalna začetna iteracija}
    Pri vsaki naslednji iteraciji je v najslabšem primeru cilj najden tik za zadnjo postajo, s čimer se prihrami manj kot $1l$. Na primer skupni stroški prvih treh iteracij skoraj $8 + 25 + 67 = 100$, cilj 
    pa je najden tik za območjem druge iteracije. Torej je konkurenčno razmerje malo manjše od $100/25 = 4$.
    \\ 
    Ko večamo zaporedje iteracij vidimo, da se razmerje med gorivom dveh zaporednih iteracij bliža k $2$, Konkurenčna razmerja pa k $4$.
\end{frame}

% -------------------------------------------------------------------
\begin{frame}
    \frametitle{Prvih 10 iteracij v izboljšani strategiji}
    \begin{table}
        \centering
        \begin{tabular}{cccccc}
        \toprule
        Iter. & Gorivo & Razdalja & Dodana razd. & Konk. raz. & Raz. goriva \\
        \midrule
        1 & 8 & 1.358929 & & 3.652759 &  \\
        2 & 25 & 1.907979 & 0.549051 & 3.987737 & 3.125000 \\
        3 & 67 & 2.394676 & 0.486697 & 3.961064 & 2.680000 \\
        4 & 168 & 2.852076 & 0.457400 & 3.986346 & 2.507463 \\
        5 & 404 & 3.289934 & 0.437858 & 3.994787 & 2.404762 \\
        6 & 944 & 3.713936 & 0.424002 & 3.997901 & 2.336634 \\
        7 & 2160 & 4.127655 & 0.413719 & 3.999123 & 2.288136 \\
        8 & 4864 & 4.533467 & 0.405812 & 3.999624 & 2.251852 \\
        9 & 10816 & 4.933022 & 0.399555 & 3.999836 & 2.223684 \\
        10 & 23808 & 5.327507 & 0.394485 & 3.999927 & 2.201183 \\
        \bottomrule
        \end{tabular}
        \end{table}
\end{frame}



% -------------------------------------------------------------------
%\begin{frame}
%    \frametitle{Optimalna začetna iteracija}
%    Prva iteracija ima eno postajo. Naslednja iteracija ima lahko $7$ postaj, saj bi v najslabšem primeru cilj lahko našli samo z dvema postajama, stroški pa ne smejo preseči 
%    $1+7=8$, saj $\frac{1+7}{8} = 4$. Tretja iteracija pa ima lahko $24$ postaj, saj je skupna cena $1+7+24=32$ in, ker lahko v najslabšem primeru dosežemo cilj z osmimi postajami, dobimo 
%    $\frac{1+7+24}{8}=4$. Če z $x_i$ označimo število postaj za $i$-to iteracijo, dobimo zvezo: 
%    $$x_n = 4(x_{n-1}+1) - \sum_{i=1}^{n-1} x_i,$$ in velja $\frac{x_n}{x_{n-1}} \searrow 2,$ ko pošljemo $n$ v neskončnost.
%\end{frame}

% -------------------------------------------------------------------
%\begin{frame}
%    \frametitle{Optimalna začetna iteracija}
%    Če pogledamo razmerje med zaporednima iteracijama:
%    $$ \frac{x_n}{x_{n-1}} $$
%\end{frame}

% -------------------------------------------------------------------   
\section{Zaključek} 
% -------------------------------------------------------------------
 \begin{frame}
    \frametitle{Literatura}
    \begin{thebibliography}{9}
        \bibitem{A jeep crostting a desert of unknown width}
        Richard E. Korf (2022) \textit{A Jeep Crossing a Desert of Unknown Width}, The
        American Mathematical Monthly, 129:5, 435-444, DOI: 10.1080/00029890.2022.2051404
        
        \bibitem{A jeep crostting a desert of unknown width (Extended Abstract)}
        Richard E. Korf (2022) \textit{A Jeep Crossing a Desert of Unknown Width (Extended Abstract)}, DOI: https://doi.org/10.1609/socs.v15i1.21791

        \bibitem{Harmonic series}
        Weisstein, Eric W, \textit{Harmonic Series}, [ogled 11. 11. 2022], dostopno na \url{https://mathworld.wolfram.com/HarmonicSeries.html}
        \end{thebibliography}
   \end{frame}
\end{document}
